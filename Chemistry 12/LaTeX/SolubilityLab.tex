\documentclass[12pt]{article}
\usepackage{fancyhdr,lastpage}
\usepackage{multicol}
\usepackage{graphicx}
\usepackage{amssymb}
\usepackage{epstopdf}
\DeclareGraphicsRule{.tif}{png}{.png}{`convert #1 `basename #1 .tif`.png}

\textwidth = 6.5 in
\textheight = 10 in
\oddsidemargin = 0.0 in
\evensidemargin = 0.0 in
\topmargin = 0.0 in
\headheight = 0.0 in
\headsep = 0.0 in
\parskip = 0.2in
\parindent = 0.0 in
\headsep = 10 mm

\pagestyle{fancy}

\lhead{Chemistry 12\\Naresh Chand}
\rhead{Steffen L. Norgren\\May 26$^{th}$, 2003}
\chead{Created using \LaTeX}

\begin{document}

\begin{center}
	\textbf{ }\\
	\LARGE{\textbf{\underline{Experiment 19D:}}}
	
	\Large{Applications of Solubility Product Principles}
\end{center}

\section{Purpose}

The purpose of this lab is to demonstrate the formation of temporarily hard water and effect of permanently hard water, as well as applying solubility product principles to determine how the hardness can be eliminated. Also, to determine the concentration of Cl$^-$ in a water sample by titrating with AgNO$_{3}$.

\section{Procedure}

See experiment 19D procedure for chemistry 12.
The only change in procedure is that this experiment is not actually performed, as we don�t have access to the necessary equipment. This lab is performed in theory and all resulting data is given.

\section{Data \& Observations}
\subsection{Temporarily Hard Water}

\textbf{Table 1:}
	\begin{tabular}{|c|c|}  % column alignment r=right l=left c=center
	\hline
	Appearance of Limewater Solution & Colourless \\
	\hline
	Appearance When CO$_{2}$ Initially Bubbled in & White Precipitate \\
	\hline
	Appearance When CO$_{2}$ Bubbled in for a Longer Time & Precipitate Re-dissolves \\
	\hline
	\end{tabular}
	
\textbf{Table 2:}
 	\begin{tabular}{|c|c|c|} 
	\textbf{Test Tube \#} & \textbf{Contents} & \textbf{Height of Lather} \\
	\hline
	A & Unboiled Solution & 1 cm \\
	\hline
	A & Boiled Solution & 5 cm \\
	\hline
	C & Distilled Water & 5 cm \\
	\end{tabular}

\pagebreak

\subsection{Permanently Hard Water}

\textbf{Table 3:}
	\begin{tabular}{|c|c|c|c|}
	\textbf{Test Tube} & \textbf{Contents} & \textbf{Appearance} & \textbf{Lather Height} \\
	\hline
	A & Distilled Water & Colourless Solution & 5 cm \\
	\hline
	C & 0.4M CaCl$_{4}$ & Colourless Solution & 1 cm \\
	\hline
	D & 0.4M MgSO$_{4}$ + 2.0M Na$_{2}$CO$_{3}$ & White Precipitate & 5 cm \\
	\hline
	E & 0.4M MgSO$_{4}$ + 2.0M Na$_{2}$CO$_{3}$ & White Precipitate & 5 cm \\
	\end{tabular}	

\subsection{Determination of Cl$^-$ Concentration in a Water Sample}

\textbf{Table 4:}
	\begin{tabular}{|c|c|c|c|}
	\textbf{[AgNO$_{3}$] = 0.100M} & \textbf{Trial 1} & \textbf{Trial 2} & \textbf{Trial 3} \textit{(If Needed)} \\
	\hline
	Sample \# & 1 & 2 & \\
	\hline
	Volume of Water Sample Used & 25.0 ml & 100.0 mL & \\
	\hline
	Initial Reading of AgNO$_{3}$ & 0.0 mL & 8.2 mL & \\
	\hline
	Final Reading of AgNO$_{3}$ & 83.5 mL & 11.3 mL & \\
	\hline
	Volume of AgNO$_{3}$ Used & 83.5 mL & 3.1 mL & \\
	\end{tabular}

\section{Calculations}
\subsection{Temporarily Hard Water}

\begin{enumerate}
\item{
	The addition of CO$_{2}$ to the limewater solution, Ca(OH)$_{2(aq)}$ resulted in the precipitation of CaCO$_{3}$ out of the solution. This is because the carbon dioxide, when dissolved in water, forms carbonic acid, H$_{2}$CO$_{3}$. Limewater neutralizes the carbonic acid and a carbonate ion is then formed. Since calcium carbonate, CaCO$_{3}$ has a very low solubility, it precipitates out of the solution.
\begin{equation}
	\mathrm{CO_{2(g)}+H_{2}O_{(l)}\rightleftharpoons H_{2}CO_{3(aq)}}
\end {equation}
\begin{equation}
\mathrm{H_{2}CO_{3(aq)}+2OH^-_{(aq)}\rightleftharpoons CO^{2-}_{3(aq)}+2H_{2}O_{(l)}}
\end{equation}
\begin{equation}
	\mathrm{Ca^{2+}_{(aq)}+CO^{2-}_{3(aq)}\rightleftharpoons CaCO_{3(s)}}
\end{equation}

These equations can be combined and simplified as the following reaction:
\begin{equation}
	\mathrm{CO_{2(g)}+Ca^{2+}_{(aq)}+2OH^-_{(aq)}\rightleftharpoons CaCO_{3(s)}+H_{2}O_{(l)}}
\end{equation}
}
\pagebreak
\item{By bubbling in more CO$_{2}$ into the solutions, we set up a new equilibrium:
\begin{equation}
	\mathrm{CO_{2(g)}+H_{2}O_{(l)}\rightleftharpoons H_{2}CO_{3(aq)}\rightleftharpoons H^+_{(aq)}+ HCO^-_{3(aq)}}
\end{equation}

Now that the H$^+$ is free, it combines with the CO$^{2-}_{3}$ from the CaCO$_{3}$ equilibrium, which was previously precipitated out of the solution. This causes the CaCO$_{3}$ to dissolve, which makes the solution become clear because of Ca(HCO$_{3}$)$_2$ formation. 

By boiling the solution of Ca(HCO$_{3}$)$_2$, the equilibrium is shifted once again as to produce CO$_2$ gas. This causes the equation to shift towards favoring the production, and subsequent precipitation of CaCO$_3$. 
}

\item{Test tube A contained the fewest soap suds because of the presence of dissolved Ca$^{2+}$ ions, which readily combine with soap to form a precipitate instead of a nice sudsy lather.
}
\end{enumerate}

\subsection{Permanently Hard Water}

\begin{enumerate}
\item{
In test tubes D and E, Na$_2$CO$_3$ was added, which resulted in the formation of a white precipitate and a higher amount of lather when mixed with soap. In test tube D, the MgSO$_4$ reacted with the Na$_2$CO$_3$ to form a white precipitate of MgCO$_3$.
\begin{equation}
	\mathrm{Mg^{2+}_{(aq)}+CO^{2-}_{3(aq)} \rightarrow MgCO_{3(s)}}
\end{equation}

In test tube E, the Na$_2$CO$_3$ reacted with the CaCl$_2$ to also form a precipitate.
\begin{equation}
	\mathrm{Ca^{2+}_{(aq)}+CO^{2-}_{3(aq)} \rightarrow CaCO_{3(s)}}
\end{equation}
}
\item{By adding soap to each test tube and shaking to form suds, we are able to differentiate between solutions that contain hard water and water that is either distilled, as in test tube A, or has had the hardness removed by adding Na$_2$CO$_3$ to the solution. The test tubes that contained the most suds were test tube A, which was distilled water, and test tubes D \& E, which had Na$_2$CO$_3$ added to remove the hardness.
}
\end{enumerate}
\pagebreak
\subsection{Determination of Chloride Ion Concentration in a Water Sample}
\begin{enumerate}
\item{
When adding AgCl to water with an unknown [Cl$^-$] concentration, the Ag$^+$ will react with any Cl$^-$ in the water, causing a precipitate of AgCl to form. The net ionic equation is as follows:
\begin{equation}
	\mathrm{Ag^+_{(aq)} + Cl^-_{(aq)} \rightarrow AgCl_{(s)}}
\end{equation}
We are able to determine when all the Cl$^-$ has been titrated out of the solution because we added Na$_2$CrO$_4$ as an indicator. The solution is yellow because of the presence of Na$_2$CrO$_4$, but when all the Cl$^-$ has been titrated out of the solution, the Ag$^+$ begins to react with the Na$_2$CrO$_4$, causing the solution to turn brick red in colour. The net ionic equation for the reaction of Ag$^+$ with Na$_2$CrO$_4$ is as follows:
\begin{equation}
\mathrm{2Ag^+_{(aq)} + CrO^{2-}_{3(aq)} \rightarrow Ag_2CrO_{3(s)}}
\end{equation}
}
\item{}
	\renewcommand{\labelenumii}{\alph{enumii}.}
	\begin{enumerate}
	\item{Trial 1:
	\begin{equation}
		\mathrm{Moles\, of\, AgNO_3\, used =0.100\frac{mol}{L}\times 0.0835\, L=8.35\times10^{-3}\, mol}
	\end{equation}
	Trial 2:
	\begin{equation}
		\mathrm{Moles\, of\, AgNO_3\, used =0.100\frac{mol}{L}\times 0.0031\, L=3.1\times10^{-4}\, mol}
	\end{equation}	
	}
	\item{
		Since $\mathrm{Ag^+_{(aq)} + Cl^-_{(aq)} \rightarrow AgCl_{(s)}}$, then we can assume that moles of Cl$^-$ equals moles of AgNO$_3$ used. Therefore, the number of moles of Cl$^-$ present for trial one is $8.35\times10^{-3}$ mol and $3.1\times10^{-4}$ mol for trial two.\\
	}
	\item{Trial 1 Cl$^-$ concentration:
	\begin{equation}
		\mathrm{[Cl^-]=\frac{8.35\times10^{-3}\, mol}{0.0250\, L}=0.334\, M}
	\end{equation}
	Trial 2 Cl$^-$ concentration:
	\begin{equation}
		\mathrm{[Cl^-]=\frac{3.1\times10^{-4}\, mol}{0.100\, L}=0.031\, M}
	\end{equation}
	}
	\pagebreak
	\item{Parts per million can be calculated using the following equation:
	\begin{equation}
		\mathrm{ppm=\frac{mass\, of\, solute}{mass\, of\, solvent}\times 10^6}
	\end{equation}
	
	Trial 1 Cl$^-$ in parts per million:
	\begin{equation}
		\mathrm{Mass\, of\, Cl^- = 8.35\times10^{-3}\, mol\times \frac{107.8682\,g}{mol}=0.901\,g}
	\end{equation}
	\begin{equation}
		\mathrm{Mass\, of\, Water = 0.0250\, L \times \frac{1000\, g}{L}=25.0\, g}
	\end{equation}
	\begin{equation}
		\mathrm{ppm\, of\, Cl^-=\frac{0.901\, g}{25.0\, g}\times 10^6=3.60\times 10^4}
	\end{equation}
	
	Trial 2 Cl$^-$ in parts per million:
	\begin{equation}
		\mathrm{Mass\, of\, Cl^- = 3.1\times10^{-4}\, mol\times \frac{107.8682\,g}{mol}=0.033\,g}
	\end{equation}
	\begin{equation}
		\mathrm{Mass\, of\, Water = 0.1000\, L \times \frac{1000\, g}{L}=100.0\, g}
	\end{equation}
	\begin{equation}
		\mathrm{ppm\, of\, Cl^-=\frac{0.033\, g}{100.0\, g}\times 10^6=3.3\times 10^2}
	\end{equation}
	}
	\end{enumerate}
\end{enumerate}
\pagebreak

\section{Questions}
\begin{enumerate}
\item{
Most caves are made from limestone, with the exception of volcanic caves, which are formed by lava flowing through the Earth's crust. Limestone can be dissolved by natural acids that are present in the groundwater. Calcium carbonate (CaCO$_3$), the principal mineral comprising limestone, is dissolved in the presence of carbonic acid (H$_{2}$CO$_{3}$) and forms calcium ions (Ca$^{2+}$ and bicarbonate ions (HCO$_3$). If the acid is able to flow through the cracks in the rock, calcium ions will be removed and a cavity or solution conduit will form.
\\
\\
The formations that hang from the ceiling of a cave are stalactites; those built up above the floor of a cave are stalagmites.  When water containing calcium ions (Ca$^{2+}$ and bicarbonate ions (HCO$_3$) drips slowly through the cave's roof, the droplets of water evaporate, leaving behind tiny amounts of calcium carbonate on the cavern ceiling. In this way, stalactites grow downward from the vaulted cavern roof, particle by particle, and stalagmites grow from the floor as any leftover water falls and evaporates from them.
}
\item{Kettle and boiler scale is the result of calcium carbonate deposits resulting from the repeated boiling of water. The scale will slowly build up as water is boiled off. The act of boiling the water in kettles and boilers, causes the Ca$^+$ and HCO$^-_3$ to be effectively removed from the water to form CaCO$_3$ deposits within the kettle or boiler.
}
\item{The certain unnamed coffee machine's parent company probably lacked a good marketing department, as they could have sold vinegar as separate product called coffee machine cleanser. In any case, the acid in the vinegar has the same effect as H$_{2}$CO$_{3}$ does in the formation of limestone caves; it dissolves the CaCO$_3$ build-up in the coffee maker, but in this case the acid is acetic acid, or CH$_3$COOH.
}
\item{Because if you were to just simply add the soap (C$_{17}$H$_{35}$COO$^-$), it would readily combine with the Ca$^{2+}$ and / or Mg$^{2+}$ ions to form what is commonly known as soap scum instead of a nice lather. When the soap precipitates out of the solution it hinders the ability of the soap to clean the clothing in the washing machine; therefore, the addition of washing soda (Na$_2$CO$_3$) causes the Ca$^{2+}$ and / or Mg$^{2+}$ ions to precipitate out and allows the soap to form a proper lather and perform its intended function.
}
\item{The release of vast quantities of phosphates, primarily calcium phosphate (Ca(H$_2$PO$_4$)$_2$), into the open environment caused a shift in the balance of life in lakes, rivers, and streams. Phosphate-loving algae flourished, removing much of the oxygen from the water due to their rapid reproduction, leaving other plant and animal life at risk. In the early '70's, limitations on the use of phosphates were implemented in the US and Canada.
}
\pagebreak
\item{
\begin{equation}
	\mathrm{moles\, AgNO_3\, used=0.10\, M \frac{mol}{L} \times 0.1\overline{5}2\, L = 0.01\overline{5}2\, M}
\end{equation}
		Since $\mathrm{Ag^+_{(aq)} + Cl^-_{(aq)} \rightarrow AgCl_{(s)}}$, then we can assume that moles of Cl$^-$ equals moles of AgNO$_3$ used. Therefore, the number of moles of Cl$^-$ present is 0.015 mol.
		
\begin{equation}
	\mathrm{[Cl^-]=\frac{0.01\overline{5}2\, mol}{0.0250\, L}=0.61\, M}
\end{equation}
}
\end{enumerate}
\section{Conclusion}
Throughout the course of this experiment we observed the methods by which we can remove the hardness of temporary and permanent hard water. In part one of the experiment we demonstrated that the hardness of temporarily hard water can be removed by by boiling the water. Because CaCO$_3$ is insoluble in water, boiling the water effectively removes the Ca$^{2+}$ present in the water. As a result, water containing HCO$^-_{3}$ and Ca$^{2+}$ / Mg$^{2+}$ is called temporarily hard water since the hardness is removed in the act of boiling the water.

Permenantly hard water is a different story, as we have to remove the offending Ca$^{2+}$ / Mg$^{2+}$ by precipitating them out through adding a water softening compound such as AgNO$_3$. We have to do this because there is no HCO$^-_{3}$ already present in the water to react with the Ca$^{2+}$ and Mg$^{2+}$ ions. Washing soda, which is simply sodium bicarbonate (Na$_2$CO$_3$), can also cause the Ca$^{2+}$ and / or Mg$^{2+}$ ions to precipitate out of the permanently hard water.

In part three of our experiment determined the Cl$^-$ concentration of two separate samples of water. We were able to determine the concentration of Cl$^-$ in the solution because we added Na$_2$CrO$_4$ as an indicator for when the AgNO$_3$ has reacted with all the Cl$^-$ in the solution. The solution is initially yellow because of the presence of Na$_2$CrO$_4$, but when all the Cl$^-$ has been titrated out of the solution, the Ag$^+$ begins to react with the Na$_2$CrO$_4$, causing the solution to turn brick red in colour. As a result of accurately measuring the amount of AgNO$_3$ added to the salt-water solution when the solution turns brick red, we consequently know the amount of Cl$^-$ that has titrated out of the solution. With this knowledge we know exactly which solution was fresh water and which was salt water. In our case, the 25.0 mL sample of water was a salt water sample as it contained $\mathrm{3.60\times 10^4\; ppm\; of\; Cl^-}$, which is quite high, and the 100.0 mL sample of water was fresh water as it contained only $\mathrm{3.3\times 10^2\; ppm\; of\; Cl^-}$.
\end{document}