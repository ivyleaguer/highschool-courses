\documentclass[12pt]{article}
\usepackage{fancyhdr,lastpage}
\usepackage{multicol}
\usepackage{graphicx}
\usepackage{amssymb}
\usepackage{epstopdf}
\DeclareGraphicsRule{.tif}{png}{.png}{`convert #1 `basename #1 .tif`.png}

\textwidth = 6.5 in
\textheight = 10 in
\oddsidemargin = 0.0 in
\evensidemargin = 0.0 in
\topmargin = 0.0 in
\headheight = 27.18335 pt
\headsep = 0.0 in
\parskip = 0.2in
\parindent = 0.0 in
\headsep = 10 mm

\pagestyle{fancy}

\lhead{Chemistry 12\\Naresh Chand}
\rhead{Steffen L. Norgren\\June 17$^{th}$, 2003}
\chead{Created using \LaTeX}

\begin{document}

\begin{center}
	\LARGE{\textbf{\underline{Experiment 20D:}}}
		
	\Large{Hydrolysis - The Reaction of Ions with Water}
\end{center}

\section{Purpose}

To identify the whether a salt has undergone hydrolysis by measuring its pH and to explain whether hydrolysis has occurred in terms of relative strengths of acids / bases from which a given salt is made. Also, to deduce which is greater for some amphiprotic anions, the K$_{a}$ for the further ionisation of the ion, or the K$_{b}$ for the hydrolysis of the ion.

\section{Procedure}

See experiment 20D procedure for chemistry 12.
The only change in procedure is that this experiment is not actually performed, as we don�t have access to the necessary equipment. This lab is performed in theory and all resulting data is given.

\section{Data \& Observations}

\textbf{Table 1:}
	\begin{tabular}{|l|c|c|c|}  % column alignment r=right l=left c=center
	Solution & Colour of Universal & pH & Type of Hydrolysis \\
	& Indicator & & \small{(Anionic, Cationic, Both, or Neither)} \\\hline
	NaCH$_3$COO & green-turquoise & 8 & Anionic \\\hline
	NaCl & yellow & 6 & Neither \\\hline
	NH$_4$Cl & orange-yellow & 5.5 & Cationic \\\hline
	(NH$_4$)$_2$SO$_4$ & orange-yellow & 5.5 & Both \\\hline
	AlCl$_3$ & red & 4 & Cationic \\\hline
	Ca(NO$_3$)$_3$ & yellow & 6 & Neither \\\hline
	Fe$_2$(SO$_4$)$_3$ & red & $\leqslant$4 & Both \\\hline
	Na$_2$CO$_3$ & deep blue & $\geqslant$10 & Anionic \\\hline
	Na$_2$PO$_4$ & deep blue & $\geqslant$10 & Anionic \\\hline
	K$_2$SO$_4$ & green & 7 & Anionic \\\hline
	KBr & yellow & 6 & Neither \\\hline
	(NH$_4$)$_2$C$_2$O$_4$ & yellow-green & 6 & Both \\\hline
	NH$_4$CH$_3$COO & yellow-green & 6.5 & Both \\\hline
	(NH$_4$)$_2$CO$_3$ & turquoise & 8.5 & Both \\\hline
	\end{tabular}

\textbf{Table 2:}
	\begin{tabular}{|l|c|c|c|}  % column alignment r=right l=left c=center
	Solution & Colour of Universal & pH & Type of Reaction (Anionic \\
	& Indicator & & \small{Hydrolysis or Further Ionization)} \\\hline
	K$_2$HPO$_4$ & blue & 9 & Hydrolysis \\\hline
	KH$_2$PO$_4$ & red & $>$4 & Ionisation \\\hline
	NaHCO$_3$ & blue & 9 & Hydrolysis \\\hline
	KHCO$_4$ & red & 4 & Ionisation \\\hline
	NaHSO$_3$ & orange-yellow & 5.5 & Ionisation \\\hline
	\end{tabular}

\section{Questions \& Calculations}
\begin{enumerate}

\item{
	NaCH$_3$COO; CH$_3$COO$^-$, CH$_3$COOH; Weak Base
	\begin{equation}
	\mathrm{CH_3CHOO^-_{(aq)} + H_2O_{(l)} \rightleftharpoons CH_3CHOOH_{(aq)} + OH^-_{(aq)}}
	\end{equation}
	
	NaCl; N/A; No Acid or Base\\
	
	NH$_4$Cl; NH$_4^+$, NH$_3$ Weak Acid
	\begin{equation}
	\mathrm{NH_{4(aq)}^ + H_2O \rightleftharpoons NH_{3(aq)} + H_3O^+_{(aq)}}
	\end{equation}
	
	(NH$_4$)$_2$SO$_4$; NH$_4^+$, NH$_3$, SO$_4^{2-}$, HSO$_4^-$ Weak Acid \& Base
	\begin{equation}
	\mathrm{2NH^+_{4(aq)} + SO^{2-}_{4(aq)} + 2H_2O_{(l)} \rightleftharpoons 2NH_{3(aq)} + HSO^-_{4(aq)} + H_3O^+_{(aq)}}
	\end{equation}
	
	AlCl$_3$; Al(H$_2$O)$_6^{3+}$, Al(H$_2$O)$_5$OH$^{2+}$; Weak Acid
	\begin{equation}
	\mathrm{Al(H_2O)^{3+}_{6(aq)} + H_2O_{(l)} \rightleftharpoons Al(H_2O)_5(OH)^{2+}_{(aq)} + H_3O^+_{(aq)}}
	\end{equation}
	
	Ca(NO$_3$)$_3$; N/A; No Acid or Base\\
	
	Fe$_2$(SO$_4$)$_3$; Fe(H$_2$O)$_6^{3+}$, Fe(H$_2$O)$_5$(OH)$^{2+}$, SO$_4^{2-}$, HSO$_4^-$; Weak Acid \& Base
	\begin{equation}
	\mathrm{2Fe(H_2O)^{3+}_{6(aq)} + SO^{2-}_{4(aq)} + 2H_2O_{(l)} \rightleftharpoons 2Fe(H_2O)_5OH^{2+}_{(aq)} + HSO^-_{4(aq)} + H_3O_{(aq)}}
	\end{equation}
	
	Na$_2$CO$_3$; HCO$_3^-$, CO$_3^{2-}$; Weak Base
	\begin{equation}
	\mathrm{CO^{2-}_{3(aq)} + H_2O_{(l)} \rightleftharpoons HCO^-_{3(aq)} + OH^-_{(aq)}}
	\end{equation}
	
\newpage

	Na$_2$PO$_4$; HPO$_4^{2-}$, PO$_4^{3-}$; Weak Base
	\begin{equation}
	\mathrm{PO^{3-}_{4(aq)} + H_2O_{(l)} \rightleftharpoons HPO^{2-}_{4(aq)} + OH^-_{(aq)}}
	\end{equation}
	
	K$_2$SO$_4$; HSO$_4^-$, SO$_4^{2-}$; Weak Base
	\begin{equation}
	\mathrm{SO^{2-}_{4(aq)} + H_2O_{(l)} \rightleftharpoons HSO^{-}_{4(aq)} + OH^-_{(aq)}}
	\end{equation}

	KBr; N/A\\
	
	(NH$_4$)$_2$C$_2$O$_4$; NH$_4^+$, NH$_3$, HC$_2$O$_4$, C$_2$O$_4^-$; Weak Acid \& Base
	\begin{equation}
	\mathrm{2NH^+_{4(aq)} + C_2O^{-}_{4(aq)} + 2H_2O_{(l)} \rightleftharpoons 2NH_{3(aq)} + HC_2O_{4(aq)} + H_3O^+_{(aq)}}
	\end{equation}
	
	NH$_4$CH$_3$COO; NH$_4^+$, NH$_3$, CH$_3$COO$^-$, CH$_3$COOH; Weak Acid \& Base
	\begin{equation}
	\mathrm{NH^+_{4(aq)} + CH_3COO^{-}_{(aq)} + H_2O_{(l)} \rightleftharpoons 2NH_{3(aq)} + CH_3COOH_{(aq)} + H_3O^+_{(aq)}}
	\end{equation}
	
	(NH$_4$)$_2$CO$_3$; NH$_4^+$, NH$_3$, HCO$_3^-$, CO$_3^{2-}$; Weak Acid \& Base
	\begin{equation}
	\mathrm{2NH^+_{4(aq)} + CO^{2-}_{3(aq)} + 2H_2O_{(l)} \rightleftharpoons 2NH_{3(aq)} + HCO^-_{3(aq)} + H_3O^+_{(aq)}}
	\end{equation}
}
\newpage

\item{
	\begin{equation}
	\mathrm{NH^+_{4(aq)} + H_2O_{(l)} \rightleftharpoons NH_{3(aq)} + H_3O^+_{(aq)}}
	\end{equation}
	\begin{equation}
	\mathrm{SO^{2-}_{4(aq)} + H_2O_{(l)} \rightleftharpoons HSO^-_{4(aq)} + OH^-_{(aq)}}
	\end{equation}
	\begin{equation}
	\mathrm{K_a(NH^+_4) = 5.6\times 10^{-10}}
	\end{equation}
	\begin{equation}
	\mathrm{K_b(SO^{2-}_4) = \frac{K_w}{K_a(HSO^{-}_4)}=\frac{1.0\times 10^{-14}}{1.2\times 10^{-2}}=8.3\times 10^{-13}}
	\end{equation}
	\begin{equation}
	\mathrm{K_a(NH^+_4)>K_b(SO^{2-}_4)}
	\end{equation}
	
	\begin{equation}
	\mathrm{NH^+_{4(aq)} + H_2O_{(l)} \rightleftharpoons NH_{3(aq)} + H_3O^+_{(aq)}}
	\end{equation}
	\begin{equation}
	\mathrm{C_2O^{-}_{4(aq)} + H_2O_{(l)} \rightleftharpoons HC_2O_{4(aq)} + OH^-_{(aq)}}
	\end{equation}
	\begin{equation}
	\mathrm{K_a(NH^+_4) = 5.6\times 10^{-10}}
	\end{equation}
	\begin{equation}
	\mathrm{K_b(C_2O^{-}_4) = \frac{K_w}{K_a(HC_2O_4)}=\frac{1.0\times 10^{-14}}{6.4\times 10^{-5}}=1.6\times 10^{-10}}
	\end{equation}
	\begin{equation}
	\mathrm{K_a(NH^+_4)>K_b(C_2O^{-}_4)}
	\end{equation}
	
	\begin{equation}
	\mathrm{NH^+_{4(aq)} + H_2O_{(l)} \rightleftharpoons NH_{3(aq)} + H_3O^+_{(aq)}}
	\end{equation}
	\begin{equation}
	\mathrm{CH_3COO^{-}_{(aq)} + H_2O_{(l)} \rightleftharpoons + CH_3COOH_{(aq)} + OH^-_{(aq)}}
	\end{equation}
	\begin{equation}
	\mathrm{K_a(NH^+_4) = 5.6\times 10^{-10}}
	\end{equation}
	\begin{equation}
	\mathrm{K_b(CH_3COO^{-}) = \frac{K_w}{K_a(CH_3COOH)}=\frac{1.0\times 10^{-14}}{6.5\times 10^{-5}}=1.5\times 10^{-10}}
	\end{equation}
	\begin{equation}
	\mathrm{K_a(NH^+_4)>K_b(CH_3COO^{-})}
	\end{equation}
	
	\begin{equation}
	\mathrm{NH^+_{4(aq)} + H_2O_{(l)} \rightleftharpoons NH_{3(aq)} + H_3O^+_{(aq)}}
	\end{equation}
	\begin{equation}
	\mathrm{CO^{2-}_{3(aq)} + H_2O_{(l)} \rightleftharpoons HCO^-_{3(aq)} + OH^-_{(aq)}}
	\end{equation}
	\begin{equation}
	\mathrm{K_a(NH^+_4) = 5.6\times 10^{-10}}
	\end{equation}
	\begin{equation}
	\mathrm{K_b(CO^{2-}_3) = \frac{K_w}{K_a(HCO^{-}_3)}=\frac{1.0\times 10^{-14}}{5.6\times 10^{-11}}=1.8\times 10^{-4}}
	\end{equation}
	\begin{equation}
	\mathrm{K_a(NH^+_4)<K_b(CO^{2-}_3)}
	\end{equation}
}
\pagebreak

\item{
	\begin{equation}
	\mathrm{HPO^{2-}_{4(aq)} + H_2O_{(l)} \rightleftharpoons PO^{3-}_{4(aq)} + H_3O^+_{(aq)}}
	\end{equation}
	\begin{equation}
	\mathrm{HPO^{2-}_{4(aq)} + H_2O_{(l)} \rightleftharpoons H_2PO^{-}_{4(aq)} + OH^-_{(aq)}}
	\end{equation}

	\begin{equation}
	\mathrm{H_2PO^{-}_{4(aq)} + H_2O_{(l)} \rightleftharpoons HPO^{2-}_{4(aq)} + H_3O^+_{(aq)}}
	\end{equation}
	\begin{equation}
	\mathrm{H_2PO^{-}_{4(aq)} + H_2O_{(l)} \rightleftharpoons H_3PO_{4(aq)} + OH^-_{(aq)}}
	\end{equation}

	\begin{equation}
	\mathrm{HCO^{-}_{3(aq)} + H_2O_{(l)} \rightleftharpoons CO^{2-}_{3(aq)} + H_3O^+_{(aq)}}
	\end{equation}
	\begin{equation}
	\mathrm{HCO^{-}_{3(aq)} + H_2O_{(l)} \rightleftharpoons H_2CO_{3(aq)} + OH^-_{(aq)}}
	\end{equation}

	\begin{equation}
	\mathrm{HSO^{-}_{4(aq)} + H_2O_{(l)} \rightleftharpoons SO^{2-}_{4(aq)} + H_3O^+_{(aq)}}
	\end{equation}
	\begin{equation}
	\mathrm{HSO^{-}_{4(aq)} + H_2O_{(l)} \rightleftharpoons H_2SO_{4(aq)} + OH^-_{(aq)}}
	\end{equation}

	\begin{equation}
	\mathrm{HSO^{-}_{3(aq)} + H_2O_{(l)} \rightleftharpoons SO^{2-}_{3(aq)} + H_3O^+_{(aq)}}
	\end{equation}
	\begin{equation}
	\mathrm{HSO^{-}_{3(aq)} + H_2O_{(l)} \rightleftharpoons H_2SO_{3(aq)} + OH^-_{(aq)}}
	\end{equation}
}
\pagebreak
\item{
For K$_2$HPO$_4$, according to the experimental results, the reaction\\ $\mathrm{HPO^{2-}_{4(aq)} + H_2O_{(l)} \rightleftharpoons H_2PO^{-}_{4(aq)} + OH^-_{(aq)}}$ occurred to a greater extent, producing a basic solution.
	\begin{equation}
	\mathrm{K_a(HPO^{2-}_4) = 2.2\times 10^{-13}}
	\end{equation}
	\begin{equation}
	\mathrm{K_a(H_2PO^{-}_4) = 6.2\times 10^{-8}}
	\end{equation}
	\begin{equation}
	\mathrm{K_b(PO^{3-}_4) = \frac{K_w}{K_a(HPO^{2-}_4)}=\frac{1.0\times 10^{-14}}{2.2\times 10^{-13}}=4.5\times 10^{-2}}
	\end{equation}
	\begin{equation}
	\mathrm{K_b(HPO^{2-}_4) = \frac{K_w}{K_a(H_2PO^{-}_4)}=\frac{1.0\times 10^{-14}}{6.2\times 10^{-8}}=1.6\times 10^{-7}}
	\end{equation}
$\mathrm{K_b(HPO^{2-}_4})$ turns out to be the strongest of all acids and bases, this is why the solution is basic.

For KH$_2$PO$_4$, according to the experimental results, the reaction\\ $\mathrm{H_2PO^{-}_{4(aq)} + H_2O_{(l)} \rightleftharpoons HPO^{2-}_{4(aq)} + H_3O^+_{(aq)}}$ occurred to a greater extent, producing an acidic solution.
	\begin{equation}
	\mathrm{K_a(H_2PO^{-}_4) = 6.2\times 10^{-8}}
	\end{equation}
	\begin{equation}
	\mathrm{K_a(H_3PO_4) = 7.5\times 10^{-3}}
	\end{equation}
	\begin{equation}
	\mathrm{K_b(HPO^{2-}_4) = \frac{K_w}{K_a(H_2PO^{-}_4)}=\frac{1.0\times 10^{-14}}{6.2\times 10^{-8}}=1.6\times 10^{-7}}
	\end{equation}
	\begin{equation}
	\mathrm{K_b(H_2PO^{-}_4) = \frac{K_w}{K_a(H_3PO_4)}=\frac{1.0\times 10^{-14}}{7.5\times 10^{-3}}=1.3\times 10^{-12}}
	\end{equation}
$\mathrm{K_a(H_3PO_4)}$ turns out to be the strongest of all acids and bases, this is why the solution is acidic.


For NaHCO$_3$, according to the experimental results, the reaction\\ $\mathrm{HCO^{-}_{3(aq)} + H_2O_{(l)} \rightleftharpoons H_2CO_{3(aq)} + OH^-_{(aq)}}$ occurred to a greater extent, producing a basic solution.
	\begin{equation}
	\mathrm{K_a(HCO^{-}_3) = 5.6\times 10^{-11}}
	\end{equation}
	\begin{equation}
	\mathrm{K_a(H_2CO_3) = 4.3\times 10^{-7}}
	\end{equation}
	\begin{equation}
	\mathrm{K_b(CO^{2-}_3) = \frac{K_w}{K_a(HCO^{-}_3)}=\frac{1.0\times 10^{-14}}{5.6\times 10^{-11}}=1.8\times 10^{-4}}
	\end{equation}
	\begin{equation}
	\mathrm{K_b(HCO^{-}_3) = \frac{K_w}{K_a(H_2CO_3)}=\frac{1.0\times 10^{-14}}{4.3\times 10^{-7}}=2.3\times 10^{-8}}
	\end{equation}
$\mathrm{K_b(CO^{2-}_3)}$ turns out to be the strongest of all acids and bases, this is why the solution is basic.
\pagebreak

For KHSO$_4$, according to the experimental results, the reaction\\ $\mathrm{HSO^{-}_{4(aq)} + H_2O_{(l)} \rightleftharpoons SO^{2-}_{4(aq)} + H_3O^+_{(aq)}}$ occurred to a greater extent, producing a acidic solution.
	\begin{equation}
	\mathrm{K_a(HSO^{-}_4) = 1.2\times 10^{-2}}
	\end{equation}
	\begin{equation}
	\mathrm{K_a(H_2SO_4) = Strong\, Acid}
	\end{equation}
	\begin{equation}
	\mathrm{K_b(SO^{2-}_4) = \frac{K_w}{K_a(HSO^{-}_4)}=\frac{1.0\times 10^{-14}}{1.2\times 10^{-2}}=8.3\times 10^{-13}}
	\end{equation}
$\mathrm{K_a(H_2SO_4)}$ is undefined because $\mathrm{H_2SO_4}$ is a strong acid, as a result the solution becomes very acidic.

For NaHSO$_3$, according to the experimental results, the reaction\\ $\mathrm{HSO^{-}_{3(aq)} + H_2O_{(l)} \rightleftharpoons SO^{2-}_{3(aq)} + H_3O^+_{(aq)}}$ occurred to a greater extent, producing a basic solution.
	\begin{equation}
	\mathrm{K_a(HSO^{-}_3) = 1.0\times 10^{-7}}
	\end{equation}
	\begin{equation}
	\mathrm{K_a(H_2SO_3) = 1.5\times 10^{-2}}
	\end{equation}
	\begin{equation}
	\mathrm{K_b(SO^{2-}_3) = \frac{K_w}{K_a(HSO^{-}_3)}=\frac{1.0\times 10^{-14}}{1.0\times 10^{-7}}=1.0\times 10^{-7}}
	\end{equation}
	\begin{equation}
	\mathrm{K_b(HSO^{-}_3) = \frac{K_w}{K_a(H_2SO_3)}=\frac{1.5\times 10^{-2}}{4.3\times 10^{-7}}=6.7\times 10^{-13}}
	\end{equation}
$\mathrm{K_a(H_2SO_3)}$ turns out to be the strongest of all acids and bases, this is why the solution is acidic.
}
\item{The precipitate observed in solutions containing Fe$^{3+}$ occurs because of the following reaction, where, in the last step, the product formed precipitates out of the solution.
	\begin{equation}
	\mathrm{Fe(H_2O)^{3+}_{6(aq)} + H_2O_{(l)} \rightleftharpoons Fe(H_2O)_5(OH)^{2+}_{(aq)} + H_3O_{(aq)}}
	\end{equation}
	\begin{equation}
	\mathrm{Fe(H_2O)_5(OH)^{2+}_{(aq)} + H_2O_{(l)} \rightleftharpoons Fe(H_2O)_5(OH)^{+}_{2(aq)} + H_3O_{(aq)}}
	\end{equation}
	\begin{equation}
	\mathrm{Fe(H_2O)_5(OH)^{+}_{2(aq)} + H_2O_{(l)} \rightleftharpoons Fe(H_2O)_5(OH)_{3(s)} + H_3O_{(aq)}}
	\end{equation}
}
\item{The precipitate observed in solutions containing Al$^{3+}$ occurs because of the following reaction, where, in the last step, the product formed precipitates out of the solution.
	\begin{equation}
	\mathrm{Al(H_2O)^{3+}_{6(aq)} + H_2O_{(l)} \rightleftharpoons Al(H_2O)_5(OH)^{2+}_{(aq)} + H_3O_{(aq)}}
	\end{equation}
	\begin{equation}
	\mathrm{Al(H_2O)_5(OH)^{2+}_{(aq)} + H_2O_{(l)} \rightleftharpoons Al(H_2O)_5(OH)^{+}_{2(aq)} + H_3O_{(aq)}}
	\end{equation}
	\begin{equation}
	\mathrm{Al(H_2O)_5(OH)^{+}_{2(aq)} + H_2O_{(l)} \rightleftharpoons Al(H_2O)_5(OH)_{3(s)} + H_3O_{(aq)}}
	\end{equation}
}
\pagebreak
\end{enumerate}

\section{Followup Questions}
\begin{enumerate}
\item{
If the farmer used $\mathrm{(NH_4)_2SO_4}$ the soil would become more acidic as a result of the hydrolysis of the $\mathrm{NH_4}$ contained in the salt. As a result, $\mathrm{KNO_3}$ would be a far better choice for keeping the soil from becoming too acidic.}

\item{
$\mathrm{Na_3PO_4}$ is a very strong basic solvent, as a result it can be quite useful as a detergent. The phosphate in the salt will readily react with earth metals such as calcium and magnesium. TSP is especially useful for stainless steel, which a chlorinated cleanser could corrode.}

\item{
Baking powder reacts with the acid-forming ingredient to produce carbon dioxide, which helps baking products to rise. The acid forming ingredient in this case is $\mathrm{Ca(H_2PO_4)_2}$, which reacts with the HCO$_3$ to form carbon dioxide gas. Bubbles of carbon dioxide become trapped in the flour mixture. The bubbles expand when they are heated and make the mixture rise.}

\item{
In order to stress the right side of the reaction, one could add some HCl to the solution, thereby causing more H$_3$O$^+$ ions to form, these ions, when in excess, would then react with the iron compound, causing the equation to shift to the left again. As a result we would end up with more iron in the solution.}

\item{
Since the solution is basic, we have HCO$^{-}_3$ ions, and NH$_3$ ions in the resulting solution. These two ions could react with each other to produce $\mathrm{NH_4HCO_3}$. As a result of the new product forming, the equation will shift to the right in order to regain equilibrium.
}

\item{
I would assume that when a vial containing ammonium carbonate is crushed, some of the ammonium ions could escape as gas into the air. Ammonia has quite a strong odour, and is quite a shock to the system when inhaled, as a result, ammonia is quite good at reviving unconscious patients. The patient wakes up because of a built in response that is part of the sympathetic nervous system, which controls the ``fight or flight'' response.
}
\end{enumerate}
\pagebreak
\section{Conclusion}

From the results of this experiment, we have been able to show that salts, which contain an acid or base, will undergo hydrolysis. The resulting solution depends on whether the salt contains an acid or base ion. When the salt contains an acid ions, such as $\mathrm{NH_4Cl}$, the the dissolution of this compound will result in an ammonium ions, which will then react with water (hydrolysis) to produce an acidic solution; the same is true for bases. In some cases we have salts that are made up from both an acid an a base, in these cases we need to calculate which of the two is stronger, the acid or the base. If the acid is stronger, we will end up with an acid solution, and conversely if the base is stronger, we will end up with a basic solution.

\end{document}