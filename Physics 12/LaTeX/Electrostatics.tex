\documentclass[12pt]{article}
\usepackage{fancyhdr,lastpage}
\usepackage{multicol}
\usepackage{graphicx}
\usepackage{amssymb}
\usepackage{epstopdf}
\DeclareGraphicsRule{.tif}{png}{.png}{`convert #1 `basename #1 .tif`.png}

\textwidth = 6.5 in
\textheight = 10 in
\oddsidemargin = 0.0 in
\evensidemargin = 0.0 in
\topmargin = 0.0 in
\headheight = 27.18335 pt
\headsep = 0.0 in
\parskip = 0.2in
\parindent = 0.0 in
\headsep = 10 mm

\pagestyle{fancy}

\lhead{Physics 12\\Rick Kidd}
\rhead{Steffen L. Norgren\\June 12$^{th}$, 2003}
\chead{Created using \LaTeX}

\begin{document}

\begin{center}
	\LARGE{\textbf{\underline{Electrostatics}}}
\end{center}

\#2
\begin{equation}
\overrightarrow{E}=\frac{F}{Q}=\frac{3.2\, N\, left}{2.4\times10^{-6}}=1.3\times10^{-6}\frac{N}{C}\, right
\end{equation}

\#4
\begin{equation}
Q=\frac{V\cdot r}{k}=\frac{(-6.4\times 10^4)(0.25\,m)}{9.00\times 10^9\frac{Nm^2}{C^2}}=1.8\times 10^{-6}C
\end{equation}

\#5
\begin{equation}
V=\overrightarrow{E}\cdot r=(1.5\times 10^4\frac{N}{C})(0.012\,m)=1.8\times10^2\,V
\end{equation}

\#7
\begin{equation}
\Delta E_k + \Delta E_p = 0 \rightarrow \Delta E_k = -\Delta E_p
\end{equation}
\begin{equation}
\frac{1}{2}mv^2=-Q\Delta V
\end{equation}
\begin{equation}
v=\sqrt{-\frac{2Q\Delta V}{m}}=\sqrt{-\frac{2(-3.2\times 10^{-19}\,C)(\frac{2000\,V}{2})}{6.6\times 10^{-27}\,kg}}=3.1\times 10^5\, \frac{m}{s}
\end{equation}

\#9\\
Rick, this one took me a while because you had a typo where you entered 180 $\mu$C, but the answer only works if that value is 18 $\mu$C.
\begin{equation}
F=k\frac{Q_1Q_2}{r^2}
\end{equation}
\begin{equation}
F_{CA}=(9.00\times10^9\frac{Nm^2}{C^2})\frac{(2.5\times10^{-6}\,C)(4.0\times10^{-5}\,C)}{(0.36\,m)^2}=6.\overline{9}44\,N
\end{equation}
\begin{equation}
F_{CB}=(9.00\times10^9\frac{Nm^2}{C^2})\frac{(2.5\times10^{-6}\,C)(1.8\times10^{-5}\,C)}{(0.12\,m)^2}=2\overline{8}.125\,N
\end{equation}
\begin{equation}
F_{net}=F_{CA}+F_{CB}=6.\overline{9}44\,N+(-2\overline{8}.125\,N)=21\,N\,away\,from\,-
\end{equation}

\pagebreak

\#10
\begin{equation}
\overrightarrow{E}=\frac{kQ}{r^2}
\end{equation}
\begin{equation}
\overrightarrow{E}_{q1}=\frac{(9.00\times10^9\frac{Nm^2}{C^2})(-2.0\times 10^-5\,C)}{0.90\,m)^2}=-2.\overline{2}22\times 10^5\frac{N}{C}
\end{equation}
\begin{equation}
\overrightarrow{E}_{q1}=\frac{(9.00\times10^9\frac{Nm^2}{C^2})(8.0\times 10^-6\,C)}{0.30\,m)^2}=8.0\times 10^5\frac{N}{C}
\end{equation}
\begin{equation}
\overrightarrow{E}_{net}=\overrightarrow{E}_{q1}+\overrightarrow{E}_{q2}
\end{equation}
\begin{equation}
\overrightarrow{E}_{net}=(-2.\overline{2}22\times 10^5\frac{N}{C})+8.0\times 10^5\frac{N}{C}=5.8\times 10^5\frac{N}{C}\,right
\end{equation}

\#12
\begin{equation}
V=\frac{kQ}{r}=\frac{(9.00\times10^9\frac{Nm^2}{C^2})(4.5\times 10^{-4}\,C)}{0.50\,m}=8.1\times 10^6\,V
\end{equation}

\#13
\begin{equation}
\Delta E_p=Q\Delta V=(1.60\times 10^{-19}\,C)(2.5\times 10^4\,V)=4.0\times 10^{-15}\,J
\end{equation}

\#14
\begin{equation}
Q=(10^{12}\,electrons)(-1.60\times 10^-19\,C)=-1.6\times 10^{-7}\,C
\end{equation}
\begin{equation}
V=\frac{kQ}{r}=\frac{(9.00\times10^9\frac{Nm^2}{C^2})(-1.6\times 10^{-7}\,C)}{0.40\,m}=-3.6\times 10^3\,V
\end{equation}
\begin{equation}
E=\frac{kQ}{r^2}=\frac{(9.00\times10^9\frac{Nm^2}{C^2})(1.6\times 10^2\,C)}{(0.40\,m)^2}=9.0\times10^3\,C
\end{equation}

\#15
\begin{equation}
E_p+E_k=0 \rightarrow E_k=-E_p
\end{equation}
\begin{equation}
E_{p}=k\frac{Q_1Q_2}{r}=\frac{(1.60\times 10^{-19}\,C)(1.60\times 10^{-19}\,C)}{(1.0\times 10^{-15}\, m)}=2.3\times 10^{-13}\,J
\end{equation}


\end{document}